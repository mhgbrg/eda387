\documentclass{article}
\usepackage[utf8]{inputenc}
\usepackage{fullpage}
\usepackage{amsmath}
\usepackage{mathtools}
\usepackage{enumerate}
\usepackage[parfill]{parskip} % New line instead of indentation for new paragraphs.
\usepackage{graphicx}
\usepackage{listings} % Allows for math inside verbatim
\usepackage{url}
\lstset{
  basicstyle=\ttfamily,
  mathescape
}
\graphicspath{{figures/}}

\title{EDA387: Computer Networks - Lab 2.1}
\author{Mats Högberg \& Juliana Siburian}
\date{\today}

\begin{document}

\maketitle

\section*{Value discovery in complete graphs}

\subsection*{With processor identifiers}

An algorithm for letting each processor $p_i$ discover its secret value $s_i$ for any value of $n$ is:

\begin{lstlisting}
    01: Do forever:
    02:   $r_i$ := read $s_{(i-1) \mod n}$
    03:   secret := read $r_{(i+1) \mod n}$
    04: End
\end{lstlisting}

Assume that the system is asynchronous, that all processors execute the same program, and that the values in the s-registers never change. Let $E = (c_0, c_1, \ldots)$ be a system execution of the above algorithm, where $c_0$ is an arbitrary starting configuration and $c_r$ is the configuration after round $r$.

After at most 2 rounds all processors will have executed line 02. On this line, $r_i$ is set to $s_{(i-1) \mod n}$. Since this is the only line where $r_i$ is updated, and $s_i$ never changes for any processor, we have that $\forall i : r_i = s_{(i-1) \mod n} \Leftrightarrow \forall i : r_{(i+1) \mod n} = s_i$ for all configurations $c_r$ where $r \geq 2$.

After at most 4 rounds all processors will also have executed line 03, in which the value of the \verb|secret| variable is set to $r_{(i+1) \mod n}$. This is the only line where the \verb|secret| variable is set, and we already proved that $\forall i : r_{(i+1) \mod n} = s_i$ for all $c_r, r \ge 2$, which means that \verb|secret| $= s_i$ for all processors $p_i$ for all configurations $c_r$ where $r \geq 4$.

Thus, we have proved both convergence and closure for the algorithm.

\subsection*{Without processor identifiers}

Let $s_{\max}$ be the size of the s-registers and $r_{\max}$ be the size of the r-registers. Assuming that $r_{\max} \geq (n-1) s_{\max}$, we can use the following algorithm for letting each processor $p_i$ discover its secret value $s_i$ in a network without processor identifiers:

\begin{lstlisting}
    01: Do forever:
    02:   $r_i$ := sum { read $s_j$ for all other processors $p_j$ }
    03:   secret := $r_k$ + $s_k$ - $r_i$, where $p_k$ is any other processor
    04: End
\end{lstlisting}

The proof for this algorithm is very similar to the proof for the previous algorithm. We assume that the system is asynchronous, that all processors execute the same program, and that the values in the secret registers never change. Let $E = (c_0, c_1, \ldots)$ be a system execution of the algorithm, where $c_0$ is an arbitrary starting configuration and $c_r$ is the configuration after round $r$.

After at most $n + 1$ rounds all processors will have executed line 02. On this line, the values in the s-registers of all other processors are read, and the sum of them are stored in $r_i$. We thus have that $r_i = \sum_j s_j - s_i$. Since this is the only line where $r_i$ is written and since the values in the s-registers never change, the values of the r-registers will never change after this round. Thus, we have that $\forall i : r_i = \sum_j s_j - s_i$ holds for all configurations $c_r$ where $r \geq n + 1$.

After at most $n + 1$ more rounds all processors will also have executed line 03. On this line, the value of the \verb|secret| variable is set to $r_k + s_k - r_i$, where $p_k$ is any other processor. We have already proved that $\forall i : r_i = \sum_j s_j - s_i$ holds for every round $c_r, r \geq n + 1$, so the value of the \verb|secret| variable for an arbitrary processor $p_i$ after round $2n + 2$ can be written as:
$$
    \text{secret} = r_k + s_k - r_i = \sum_j s_j - s_k + s_k - \Big(\sum_j s_j - s_i\Big) = s_i
$$

And since this is the only line where \verb|secret| is updated, we have that \verb|secret| $ = s_i$ for all processors $p_i$ for all configurations $c_r$ where $r \geq 2n + 2$.

Thus, we have proved both convergence and closure for the algorithm.

Note that in the problem statement it says that the sizes of the s and r-registers are constant and independent of $n$, but here we assume that $r_{\max} \geq (n-1) s_{\max}$. We haven't been able to come up with a solution for when this does not hold, but we haven't been able to prove that no such solution exists either.

\end{document}
